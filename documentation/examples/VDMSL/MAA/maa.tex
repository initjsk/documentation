% @(#)Appendix-B-ISO8731-2.IFAD.vdm	1.2
% 1/28/97
%\documentstyle[a4wide,vdmsl]{article}
%\pagestyle{empty}
\documentclass{article}
\usepackage{a4}
\usepackage{makeidx}
\usepackage{vdmsl-2e}

\setlength{\parindent}{0in}        % Paragraph indent is 0
\setlength{\parskip}{0.1cm}        % Paragraph separation

\nolinenumbering                    %Turn line numbering off
\setindent{Rindent}{0in}
\setindent{outer}{0in}
\setindent{inner}{\parindent}

\begin{document}
\begin{center}
\section*{{Annex B} \\
  {\normalsize (Informative)} \\
          {\ } \\
  {Specification of MAA in VDM} \\
          {\ }} 
\end{center}
\subsection*{B.0 Introduction}
 
In the following section is a complete specification of the MAA in the
specification language called the Vienna Development Method (VDM). The
notation  for the VDM is that of the emerging standard for VDM as described
in {\sl VDM Specification Language Proto-Standard, 
ISO/IEC JTC1/SC22/WG19, Document Reference IN9}.
  
It demonstrates how it is possible to write a standard in an unambiguous
formal language.
The style of the VDM has been guided by the following:
\begin{itemize}
\item It has been written in a functional way so that it could be
      implemented easily although not necessarily efficiently
      in a functional, logic or imperative programming language.
\item It retains as much of the naming, structure etc. used in the
      main part of this standard.
\end{itemize}

The VDM in the next section is written purely in VDM, including the comments.
The comments point to sections of the main text of the standard from which
the VDM is derived.
 The VDM models a message as a sequence of natural numbers
$0$ and $1$ ($Bits$).

% @(#)Appendix-B-ISO8731-2.IFAD.vdm	1.2
% 1/28/97
%\documentstyle[a4wide,vdmsl]{article}
%\pagestyle{empty}
\documentclass{article}
\usepackage{a4}
\usepackage{makeidx}
\usepackage{vdmsl-2e}

\setlength{\parindent}{0in}        % Paragraph indent is 0
\setlength{\parskip}{0.1cm}        % Paragraph separation

\nolinenumbering                    %Turn line numbering off
\setindent{Rindent}{0in}
\setindent{outer}{0in}
\setindent{inner}{\parindent}

\begin{document}
\begin{center}
\section*{{Annex B} \\
  {\normalsize (Informative)} \\
          {\ } \\
  {Specification of MAA in VDM} \\
          {\ }} 
\end{center}
\subsection*{B.0 Introduction}
 
In the following section is a complete specification of the MAA in the
specification language called the Vienna Development Method (VDM). The
notation  for the VDM is that of the emerging standard for VDM as described
in {\sl VDM Specification Language Proto-Standard, 
ISO/IEC JTC1/SC22/WG19, Document Reference IN9}.
  
It demonstrates how it is possible to write a standard in an unambiguous
formal language.
The style of the VDM has been guided by the following:
\begin{itemize}
\item It has been written in a functional way so that it could be
      implemented easily although not necessarily efficiently
      in a functional, logic or imperative programming language.
\item It retains as much of the naming, structure etc. used in the
      main part of this standard.
\end{itemize}

The VDM in the next section is written purely in VDM, including the comments.
The comments point to sections of the main text of the standard from which
the VDM is derived.
 The VDM models a message as a sequence of natural numbers
$0$ and $1$ ($Bits$).

% @(#)Appendix-B-ISO8731-2.IFAD.vdm	1.2
% 1/28/97
%\documentstyle[a4wide,vdmsl]{article}
%\pagestyle{empty}
\documentclass{article}
\usepackage{a4}
\usepackage{makeidx}
\usepackage{vdmsl-2e}

\setlength{\parindent}{0in}        % Paragraph indent is 0
\setlength{\parskip}{0.1cm}        % Paragraph separation

\nolinenumbering                    %Turn line numbering off
\setindent{Rindent}{0in}
\setindent{outer}{0in}
\setindent{inner}{\parindent}

\begin{document}
\begin{center}
\section*{{Annex B} \\
  {\normalsize (Informative)} \\
          {\ } \\
  {Specification of MAA in VDM} \\
          {\ }} 
\end{center}
\subsection*{B.0 Introduction}
 
In the following section is a complete specification of the MAA in the
specification language called the Vienna Development Method (VDM). The
notation  for the VDM is that of the emerging standard for VDM as described
in {\sl VDM Specification Language Proto-Standard, 
ISO/IEC JTC1/SC22/WG19, Document Reference IN9}.
  
It demonstrates how it is possible to write a standard in an unambiguous
formal language.
The style of the VDM has been guided by the following:
\begin{itemize}
\item It has been written in a functional way so that it could be
      implemented easily although not necessarily efficiently
      in a functional, logic or imperative programming language.
\item It retains as much of the naming, structure etc. used in the
      main part of this standard.
\end{itemize}

The VDM in the next section is written purely in VDM, including the comments.
The comments point to sections of the main text of the standard from which
the VDM is derived.
 The VDM models a message as a sequence of natural numbers
$0$ and $1$ ($Bits$).

% @(#)Appendix-B-ISO8731-2.IFAD.vdm	1.2
% 1/28/97
%\documentstyle[a4wide,vdmsl]{article}
%\pagestyle{empty}
\documentclass{article}
\usepackage{a4}
\usepackage{makeidx}
\usepackage{vdmsl-2e}

\setlength{\parindent}{0in}        % Paragraph indent is 0
\setlength{\parskip}{0.1cm}        % Paragraph separation

\nolinenumbering                    %Turn line numbering off
\setindent{Rindent}{0in}
\setindent{outer}{0in}
\setindent{inner}{\parindent}

\begin{document}
\begin{center}
\section*{{Annex B} \\
  {\normalsize (Informative)} \\
          {\ } \\
  {Specification of MAA in VDM} \\
          {\ }} 
\end{center}
\subsection*{B.0 Introduction}
 
In the following section is a complete specification of the MAA in the
specification language called the Vienna Development Method (VDM). The
notation  for the VDM is that of the emerging standard for VDM as described
in {\sl VDM Specification Language Proto-Standard, 
ISO/IEC JTC1/SC22/WG19, Document Reference IN9}.
  
It demonstrates how it is possible to write a standard in an unambiguous
formal language.
The style of the VDM has been guided by the following:
\begin{itemize}
\item It has been written in a functional way so that it could be
      implemented easily although not necessarily efficiently
      in a functional, logic or imperative programming language.
\item It retains as much of the naming, structure etc. used in the
      main part of this standard.
\end{itemize}

The VDM in the next section is written purely in VDM, including the comments.
The comments point to sections of the main text of the standard from which
the VDM is derived.
 The VDM models a message as a sequence of natural numbers
$0$ and $1$ ($Bits$).

\include{maa.vdmsl}

\end{document}


\end{document}


\end{document}


\end{document}
