\item[Semantics of Operators:] \mbox{}

\begin{TypeSemantics}
Membership & tests if {\tt e} is a member of the set {\tt s1} \\\hline

Not membership & tests if {\tt e} is not a member of the set {\tt s1} \\\hline

Union & yields the union of the sets {\tt s1} and {\tt s2}, i.e.\ the
set containing all the elements of both {\tt s1} and {\tt s2}. \\ 
\hline

Intersection & yields the intersection of sets {\tt s1} and {\tt s2},
i.e.\ the set containing the elements that are in both {\tt s1} and
{\tt s2}. \\ \hline

Difference & yields the set containing all the elements from {\tt s1}
that are not in {\tt s2}. {\tt s2} need not be a subset of {\tt s1}. \\ \hline

Subset & tests if {\tt s1} is a subset of {\tt s2}, i.e.\ whether all
elements from {\tt s1} are also in {\tt s2}. Notice that any set is a
subset of itself. \\ \hline

Proper subset & tests if {\tt s1} is a proper subset of {\tt s2}, i.e.\ it is
a subset and {\tt s2$\backslash$s1} is non-empty. \\ \hline

Cardinality & yields the number of elements in {\tt s1}. \\ \hline

Distributed union & the resulting set is the union of all the elements
(these are sets themselves) of {\tt ss}, i.e.\ it contains all the
elements of all the elements/sets of {\tt ss}. \\ \hline

Distributes intersection & the resulting set is the intersection of
all the elements (these are sets themselves) of, i.e.\ it contains
the elements that are in all the elements/sets of {\tt ss}. {\tt ss}
must be non-empty. \\ \hline

Finite power set & yields the power set of {\tt s1}, i.e.\ the set of
all subsets of {\tt s1}. \\ \hline
\end{TypeSemantics}

