\documentclass[11pt]{article}
\usepackage{a4}
\usepackage{makeidx}
%\usepackage{vdmsl-2e}
%\usepackage{color}
\usepackage{overture}
\usepackage{termref}
\usepackage{epsf}
\usepackage{latexsym}
\usepackage{longtable}

\newcommand{\StateDef}[1]{{\bf #1}}
\newcommand{\TypeDef}[1]{{\bf #1}}
\newcommand{\TypeOcc}[1]{{\it #1}}
\newcommand{\FuncDef}[1]{{\bf #1}}
\newcommand{\FuncOcc}[1]{#1}
\newcommand{\ModDef}[1]{{\tiny #1}}

\makeindex
%\documentstyle[epsf,a4,11pt,vdmsl_v1.1.31,makeidx,termref]{article}
\newcommand{\figdir}{/home/peter/toolbox}
\newcommand{\SA}{{\small SA}}
\newcommand{\SD}{{\small SD}}
\newcommand{\SASD}{{\small SA/SD}}
\newcommand{\VDM}{{\small VDM}}
\newcommand{\VDMSL}{{\small VDM-SL}}
\newcommand{\DFD}{{\small DFD}}
\newcommand{\DFDs}{{\small DFD}s}

\newcommand{\makefigure}[3]{\begin{figure}[ht]
{\leavevmode
\centering
\epsfbox{\figdir/#1.eps}
\caption{#2}\label{#3}}
\end{figure}}

\newcommand{\documenttype}{document}
\makeindex
\begin{document}

\title{A Formal Semantics of Data Flow Diagrams}
\author{Peter Gorm Larsen, Nico Plat and Hans Toetenel}
\date{November 1993}
\maketitle
\begin{abstract}
This document presents a full version of the
formal semantics of data flow diagrams reported in \cite{Larsen&93b}.
Data Flow Diagrams are used in Structured Analysis and are
 based on an abstract model for data flow transformations.
The semantics consists of a collection of \VDM\ functions,
transforming an abstract syntax representation of a data flow
diagram into an abstract syntax representation of a \VDM\
specification. Since this transformation is executable, it
becomes possible to provide a software analyst/designer with two `views' of
the system being modeled: a graphical view in terms of a data
flow diagram, and a textual view in terms of a \VDM\
specification. The specification presented in this document have been
processed by The IFAD VDM-SL Toolbox \cite{Lassen93} and the \LaTeX\ output is
produced directly by means of this tool. The complete transformation has
been syntax-checked, type-checked and tested using the
{\small IFAD VDM-SL}\ Toolbox \cite{Lassen93}; this has given us
confidence that the transformation we have defined is a reasonable one.
\end{abstract}
\newpage
\tableofcontents
\newpage
\section{Introduction}

The introduction of formal methods in industrial organizations
may become easier if these methods can be used alongside the
more widely used conventional techniques for software development,
such as `structured methods'.
Structured methods are methods for software analysis and design,
based on the use of heuristics for making analysis and design decisions.
They provide a relatively well-defined path, often in a cookbook-like
fashion (hence the term `structured' methods), starting from the analysis
of software requirements and ending at system coding.
The design notations used are usually graphical and have no formal basis.
In that sense structured and formal methods can be regarded as complementary.
It is often suggested that the informal graphical notations as provided by
structured methods are intuitively appealing to software analysts/designers.
Therefore, a combined structured/formal method may not only increase the
understanding of the use of formal methods in the software process, but 
also may increase the acceptability of formal methods to these people.

Our work in this area so far has concentrated on combining {\em Structured
Analysis (SA)} \cite{Yourdon75,Demarco78,Gane77}
with the {\em Vienna Development Method (VDM)}
\cite{Bjorner&82b,Jones90a}; we provide a brief introduction to
\SA, but we refer to text books such as \cite{Jones90a} and
\cite{Andrews&91} for an introduction to \VDM.
We think that a well-integrated combination of notations can be achieved
by using {\em data flow diagrams (DFDs)} -- which we consider to be the main
design notation of \SA\ --
as a graphical view of the system and \VDM\ as a textual
view. These different views emphasize different aspects of the
specified system: the \DFD\ graphical view
focuses on an overview of the {\em structure} of the system, whereas
the \VDM\ textual view focuses on the detailed {\em functionality} of the system.
The base of a combined structured/formal method consists of a formally defined
relation between the structured method and the formal method.
In \cite{Plat&91a} we describe several approaches to modeling \DFDs\ 
using the \VDMSL\ specification language \cite{BSIVDM92b,Dawes91}.
In this paper we discuss one such particular model in more detail,
thus essentially providing a `formal semantics' of \DFDs.
A discussion on the methodological aspects of the approach can be found in
\cite{Larsen&91b}.

The remainder of this paper is organized as follows.
In the following section a brief introduction to \SA\ is given, focusing on the
use of \DFDs.
In Section~\ref{sec:motivation} we describe our strategy for
transforming \DFDs\ into \VDM\ specifications, paying attention to the
limitations of our approach.
The main part of this paper is Section~\ref{transsec}, in which the formal
transformation from \DFDs\ to \VDM\ is presented.
Emphasis is put on the motivation for the choices
made in the transformation. The main aspects of the transformation
itself are described using \VDM\ functions together
with a number of examples.
In Section~\ref{sec:conclusions} we give an overview of related
work on formal semantics for \DFDs, and present some conclusions and
ideas for future work in this area.
This is followed by a list of references, and two appendices and an
index for the complete specification.
In the first appendix, we describe the abstract syntax representation
of \DFDs. In the second appendix we present the abstract syntax 
representation we use for the generated
\VDM\ specifications. This is a subset of the one used
in \cite{BSIVDM92b}\footnote{%
To be precise, the abstract syntax used for \VDM\ specifications is
a partof the one
called the `Outer Abstract Syntax' in \cite{BSIVDM92b}; a lack of knowledge about
this Outer Abstract Syntax
does not affect the understanding of this document, however.}).

\section{Overview of Structured Analysis}
\label{sec:sa}

{\em Structured Analysis ({\small\it SA})}
\cite{Yourdon75,Demarco78,Gane77} is one of
the most widely used methods for software analysis.
Often it is used in combination with {\em Structured Design ({\small\it SD})}
\cite{Yourdon75}; the resulting combination is called {\em \SASD}.
The approach to analysis taken in \SA\ is to concentrate on the functions
to be carried out by the system, using data flow abstraction to
describe the flow of data through a network of transforming
processes, called data transformers, together with access to data stores.
Such a network, which is the most important design product of \SA,
is called a {\em data flow diagram (DFD)}. The original version of
\SA\ was meant to be used to model sequential systems.  
A \DFD\ is a {\em directed graph} consisting of elementary building blocks.
Each building block has a graphical notation (figure~\ref{figexternalprocess}).

%\makefigure{building_blocks}{Elementary building blocks of a DFD}{figexternalprocess}

Through the years several dialects have evolved and extensions
have been defined (e.g. {\small SSADM}\ \cite{Longworth&86} and {\small SA/RT}\
\cite{Ward85}), 
but we limit ourselves to \DFDs\ with a sequential model and
a small number of building blocks:
\begin{itemize}
\item
{\em Data transformers}. Data transformers denote a transformation from
(an arbitrary number of) input values to
(an arbitrary number of) output values, possibly with side effects.
\item
{\em Data flows}. Data flows are represented as arrows, connecting one
data transformer to another. They represent a flow of data between the
data transformers they connect.
The flow of data is unidirectional in the direction of the arrow.
\item
{\em Data stores}. Data stores provide for (temporary) storage of data.
\item
{\em External processes.}
External processes are processes that are not part of the system but belong
to the outside world. They are used to show where the input to the system
is coming from and where the output of the system is going to.
\end{itemize}

\DFD s are used to model the {\em information flow} through a system.
As such they provide a limited view of the system:
in their most rudimentary form they neither show the control flow of the system
nor any timing aspects.
Therefore, \DFDs\ are often combined with data dictionaries,
control flow diagrams, state transition diagrams, decision tables
and mini-specifications to provide a comprehensive view of all the 
aspects of the system.

The process of constructing a \DFD\ is an iterative process.
Initially, the system to be designed is envisaged as one large data transformer,
getting input from and providing output to external processes.
This initial, high-level \DFD\ is called a {\em context diagram}.
The next step is the {\em decomposition}
of the context diagram into a network of data transformers,
the total network providing the same functionality as the original
context diagram.
This process is repeated for each data transformer until the analyst/designer
considers all the data transformers in the \DFD\ to be primitive,
i.e. each data transformer performs a simple operation that does not
need to be further decomposed.
We call such a collection of \DFDs, describing the same system but at different
levels of abstraction, a {\em hierarchy of DFDs}.

\section{Approach to the Transformation}
\label{sec:motivation}

Before presenting the formal transformation from \DFDs\ to \VDM\ we first explain
the underlying strategy for the transformation and the limitations
imposed upon the \DFDs\ to make our transformation valid.

\subsection{Underlying strategy}

The starting point for our transformation is the work presented in
\cite{Plat&91a}, in which the general properties of two transformations from
DFDs to \VDM\ constructs are discussed.
The main difference between these two transformations is the way data flows are
modeled: in the first transformation they are modeled as (infinitely large)
queues, in the second transformation they are modeled as operations combining
the two data transformers connected by the data flow.
The advantage of the latter transformation is that a more abstract
interpretation of \DFDs\ can be achieved, because the transformation
solely focuses on modeling the information flow through a \DFD.
This is also the reason for choosing this transformation 
as the basis for the transformation described in this paper.
One simplification with respect to the transformation described in
\cite{Plat&91a} is that the latter is
more general because the order
in which the `underlying' operations are called is left unspecified
(i.e. it is loosely specified),
which makes the operation modeling the data flow rather complicated.
In this paper, however, we are dealing with purely sequential systems,
and therefore we can assume that data flows between two
data transformers are `direct' in the sense that 
the data transformer that uses the data flow as input cannot be
called before the data transformer that uses the data flow as output.

\subsection{Transformation of DFD building blocks}


When providing a formal semantics for \DFDs\ it is
important to decide whether the \DFD\ is intended to model
a concurrent system or a sequential system. More recent versions
of \SA\ (like {\em SA/RT} \cite{Ward85}) include concurrency and can be
used to develop real-time systems. However, originally \SA\
was intended for the development of information systems
implemented in traditional imperative programming languages.
In that situation it seems natural to interpret the data transformers
as {\em functions} or
{\em operations} which, given input data, sequentially perform computations and
produce output data.
If the data flow diagrams are used to model a concurrent system
it is more natural to interpret data transformers as {\em processes},
possibly executing in parallel.
Since we restrict ourselves to sequential systems we model data transformers
as {\em VDM operations}\footnote{Data transformers neither having access to
data stores nor being connected to external processes can also be modeled
as {\em VDM functions}. In our approach only {\tiny VDM}
operations are used because
we want each different type of construct in a {\tiny DFD} to be mapped to 
(semantically) the same construct in {\tiny VDM}.
{\tiny VDM} functions and {\tiny VDM} operations
(without side-effects) semantically differ in the way {\em looseness} is interpreted
(see \cite{IFIP}).}.

To ensure that the structure of the \VDM\ specification
resembles the structure of the \DFD,
we group the operations modeling data transformers at the same level in
a hierarchy of \DFDs\ together in `modules'\footnote{%
{\tiny VDM-SL} as described in \cite{BSIVDM92b} has no structuring mechanism.
The structuring mechanism we used is based on a proposal by Bear
\cite{Bear88}. The constructs we use are simple so that an intuitive
interpretation suffices.}
importing the necessary types and
operations needed for the data transformers
(figure~\ref{hierarchy}).

%\makefigure{hierarchy}{Transformation of a hierarchy of DFDs into a VDM module structure (example)}{hierarchy}


{\em External processes} can be considered as processes `executing' in parallel with
the specified system.
In our approach we model the data flows from and to external processes
as {\em state components} in the \VDM\ specification.
This is a minor difference with the transformation presented in
\cite{Plat&91a},
in which external processes are regarded as part of the system and
are therefore modeled as \VDM\ operations in the same way as data transformers.


{\em Data stores} are modeled as \VDM\ {\em state components}.
This corresponds to the fact that data transformers
(which can be used to access and change data stores)
are modeled as \VDM\ operations,
the constructs in \VDMSL\ that can access and change state components.


We envisage {\em data flows} as constructs which can combine
two data transformers by providing communication facilities between
these two data transformers.
A data flow is, therefore, modeled as an operation
calling the operations that model the two data transformers connected by
the data flow. 
In this way a process of combining data transformers can be started during
which in each step two data transformers (connected by a data flow) are
integrated into a higher level data transformer, finally resulting in
the context diagram.

Generalizing this approach, we have chosen to combine 
all the data transformers in a \DFD\ into a
higher level data transformer in {\em one} step.
The data transformer constructed in this way is modeled as a \VDM\ operation.

\subsection{Limitations imposed upon the DFDs}

Besides restricting the expressibility of the kind of \DFDs\
for which we are able to provide semantics to sequential systems, we
assume that:

\begin{itemize}
\item Data flows not connected
      to an external process must form an acyclic
      graph at each level in the hierarchy of \DFDs. This
      is necessary because in our transformation we provide both
	  explicit \VDM\ specifications as well as implicit \VDM\ specifications
	  as models for \DFDs. Allowing general cyclic \DFDs\ would make the 
	  transformation into an explicit \VDM\ specification impossible.
	  The restriction furthermore simplifies the
      transformation of \DFDs\ into implicit \VDM\ specifications.
      In Section~\ref{depend} we come back to this restriction in more
	  detail.
\item There is a one-to-one mapping between the
      input to the system and the output from the system.
	  One-to-many mappings and many-to-one mappings are a common
	  problem when interpreting \DFDs, described in more detail
	  in \cite{Alabiso88}\footnote{%
      In \cite{Alabiso88} this problem is called {\em I/O uncohesiveness}.
      I/O uncohesiveness occurs if either a data transformer
      must consume several pieces of input data before generating
      output data, or if a data transformer generates pieces of output
      independently of all other inputs and outputs.
      Alabiso describes a solution called `the burial method',
	  centered around the generation of terminator symbols
      which indicate that `something is missing'.}.
	  However, we are not entirely satisfied with the solution proposed by
	  Alabiso, and since in our experience most of the \DFDs\ with
	  one-to-many or many-to-one mappings should be regarded as design
	  products and not as specification products, we feel that a
	  restriction to one-to-one mappings is not a serious one for our
	  purpose.
      Alternatively, the analyst may supply a mini-specification for 
      each non-primitive data transformer not obeying the restriction
	  of a one-to-one mapping between input and output.

\item To simplify the formal description the
data flows must have unique names at each level in the hierarchy
of \DFDs.
\end{itemize}

\documentclass[11pt]{article}
\usepackage{a4}
\usepackage{makeidx}
%\usepackage{vdmsl-2e}
%\usepackage{color}
\usepackage{overture}
\usepackage{termref}
\usepackage{epsf}
\usepackage{latexsym}
\usepackage{longtable}

\newcommand{\StateDef}[1]{{\bf #1}}
\newcommand{\TypeDef}[1]{{\bf #1}}
\newcommand{\TypeOcc}[1]{{\it #1}}
\newcommand{\FuncDef}[1]{{\bf #1}}
\newcommand{\FuncOcc}[1]{#1}
\newcommand{\ModDef}[1]{{\tiny #1}}

\makeindex
%\documentstyle[epsf,a4,11pt,vdmsl_v1.1.31,makeidx,termref]{article}
\newcommand{\figdir}{/home/peter/toolbox}
\newcommand{\SA}{{\small SA}}
\newcommand{\SD}{{\small SD}}
\newcommand{\SASD}{{\small SA/SD}}
\newcommand{\VDM}{{\small VDM}}
\newcommand{\VDMSL}{{\small VDM-SL}}
\newcommand{\DFD}{{\small DFD}}
\newcommand{\DFDs}{{\small DFD}s}

\newcommand{\makefigure}[3]{\begin{figure}[ht]
{\leavevmode
\centering
\epsfbox{\figdir/#1.eps}
\caption{#2}\label{#3}}
\end{figure}}

\newcommand{\documenttype}{document}
\makeindex
\begin{document}

\title{A Formal Semantics of Data Flow Diagrams}
\author{Peter Gorm Larsen, Nico Plat and Hans Toetenel}
\date{November 1993}
\maketitle
\begin{abstract}
This document presents a full version of the
formal semantics of data flow diagrams reported in \cite{Larsen&93b}.
Data Flow Diagrams are used in Structured Analysis and are
 based on an abstract model for data flow transformations.
The semantics consists of a collection of \VDM\ functions,
transforming an abstract syntax representation of a data flow
diagram into an abstract syntax representation of a \VDM\
specification. Since this transformation is executable, it
becomes possible to provide a software analyst/designer with two `views' of
the system being modeled: a graphical view in terms of a data
flow diagram, and a textual view in terms of a \VDM\
specification. The specification presented in this document have been
processed by The IFAD VDM-SL Toolbox \cite{Lassen93} and the \LaTeX\ output is
produced directly by means of this tool. The complete transformation has
been syntax-checked, type-checked and tested using the
{\small IFAD VDM-SL}\ Toolbox \cite{Lassen93}; this has given us
confidence that the transformation we have defined is a reasonable one.
\end{abstract}
\newpage
\tableofcontents
\newpage
\section{Introduction}

The introduction of formal methods in industrial organizations
may become easier if these methods can be used alongside the
more widely used conventional techniques for software development,
such as `structured methods'.
Structured methods are methods for software analysis and design,
based on the use of heuristics for making analysis and design decisions.
They provide a relatively well-defined path, often in a cookbook-like
fashion (hence the term `structured' methods), starting from the analysis
of software requirements and ending at system coding.
The design notations used are usually graphical and have no formal basis.
In that sense structured and formal methods can be regarded as complementary.
It is often suggested that the informal graphical notations as provided by
structured methods are intuitively appealing to software analysts/designers.
Therefore, a combined structured/formal method may not only increase the
understanding of the use of formal methods in the software process, but 
also may increase the acceptability of formal methods to these people.

Our work in this area so far has concentrated on combining {\em Structured
Analysis (SA)} \cite{Yourdon75,Demarco78,Gane77}
with the {\em Vienna Development Method (VDM)}
\cite{Bjorner&82b,Jones90a}; we provide a brief introduction to
\SA, but we refer to text books such as \cite{Jones90a} and
\cite{Andrews&91} for an introduction to \VDM.
We think that a well-integrated combination of notations can be achieved
by using {\em data flow diagrams (DFDs)} -- which we consider to be the main
design notation of \SA\ --
as a graphical view of the system and \VDM\ as a textual
view. These different views emphasize different aspects of the
specified system: the \DFD\ graphical view
focuses on an overview of the {\em structure} of the system, whereas
the \VDM\ textual view focuses on the detailed {\em functionality} of the system.
The base of a combined structured/formal method consists of a formally defined
relation between the structured method and the formal method.
In \cite{Plat&91a} we describe several approaches to modeling \DFDs\ 
using the \VDMSL\ specification language \cite{BSIVDM92b,Dawes91}.
In this paper we discuss one such particular model in more detail,
thus essentially providing a `formal semantics' of \DFDs.
A discussion on the methodological aspects of the approach can be found in
\cite{Larsen&91b}.

The remainder of this paper is organized as follows.
In the following section a brief introduction to \SA\ is given, focusing on the
use of \DFDs.
In Section~\ref{sec:motivation} we describe our strategy for
transforming \DFDs\ into \VDM\ specifications, paying attention to the
limitations of our approach.
The main part of this paper is Section~\ref{transsec}, in which the formal
transformation from \DFDs\ to \VDM\ is presented.
Emphasis is put on the motivation for the choices
made in the transformation. The main aspects of the transformation
itself are described using \VDM\ functions together
with a number of examples.
In Section~\ref{sec:conclusions} we give an overview of related
work on formal semantics for \DFDs, and present some conclusions and
ideas for future work in this area.
This is followed by a list of references, and two appendices and an
index for the complete specification.
In the first appendix, we describe the abstract syntax representation
of \DFDs. In the second appendix we present the abstract syntax 
representation we use for the generated
\VDM\ specifications. This is a subset of the one used
in \cite{BSIVDM92b}\footnote{%
To be precise, the abstract syntax used for \VDM\ specifications is
a partof the one
called the `Outer Abstract Syntax' in \cite{BSIVDM92b}; a lack of knowledge about
this Outer Abstract Syntax
does not affect the understanding of this document, however.}).

\section{Overview of Structured Analysis}
\label{sec:sa}

{\em Structured Analysis ({\small\it SA})}
\cite{Yourdon75,Demarco78,Gane77} is one of
the most widely used methods for software analysis.
Often it is used in combination with {\em Structured Design ({\small\it SD})}
\cite{Yourdon75}; the resulting combination is called {\em \SASD}.
The approach to analysis taken in \SA\ is to concentrate on the functions
to be carried out by the system, using data flow abstraction to
describe the flow of data through a network of transforming
processes, called data transformers, together with access to data stores.
Such a network, which is the most important design product of \SA,
is called a {\em data flow diagram (DFD)}. The original version of
\SA\ was meant to be used to model sequential systems.  
A \DFD\ is a {\em directed graph} consisting of elementary building blocks.
Each building block has a graphical notation (figure~\ref{figexternalprocess}).

%\makefigure{building_blocks}{Elementary building blocks of a DFD}{figexternalprocess}

Through the years several dialects have evolved and extensions
have been defined (e.g. {\small SSADM}\ \cite{Longworth&86} and {\small SA/RT}\
\cite{Ward85}), 
but we limit ourselves to \DFDs\ with a sequential model and
a small number of building blocks:
\begin{itemize}
\item
{\em Data transformers}. Data transformers denote a transformation from
(an arbitrary number of) input values to
(an arbitrary number of) output values, possibly with side effects.
\item
{\em Data flows}. Data flows are represented as arrows, connecting one
data transformer to another. They represent a flow of data between the
data transformers they connect.
The flow of data is unidirectional in the direction of the arrow.
\item
{\em Data stores}. Data stores provide for (temporary) storage of data.
\item
{\em External processes.}
External processes are processes that are not part of the system but belong
to the outside world. They are used to show where the input to the system
is coming from and where the output of the system is going to.
\end{itemize}

\DFD s are used to model the {\em information flow} through a system.
As such they provide a limited view of the system:
in their most rudimentary form they neither show the control flow of the system
nor any timing aspects.
Therefore, \DFDs\ are often combined with data dictionaries,
control flow diagrams, state transition diagrams, decision tables
and mini-specifications to provide a comprehensive view of all the 
aspects of the system.

The process of constructing a \DFD\ is an iterative process.
Initially, the system to be designed is envisaged as one large data transformer,
getting input from and providing output to external processes.
This initial, high-level \DFD\ is called a {\em context diagram}.
The next step is the {\em decomposition}
of the context diagram into a network of data transformers,
the total network providing the same functionality as the original
context diagram.
This process is repeated for each data transformer until the analyst/designer
considers all the data transformers in the \DFD\ to be primitive,
i.e. each data transformer performs a simple operation that does not
need to be further decomposed.
We call such a collection of \DFDs, describing the same system but at different
levels of abstraction, a {\em hierarchy of DFDs}.

\section{Approach to the Transformation}
\label{sec:motivation}

Before presenting the formal transformation from \DFDs\ to \VDM\ we first explain
the underlying strategy for the transformation and the limitations
imposed upon the \DFDs\ to make our transformation valid.

\subsection{Underlying strategy}

The starting point for our transformation is the work presented in
\cite{Plat&91a}, in which the general properties of two transformations from
DFDs to \VDM\ constructs are discussed.
The main difference between these two transformations is the way data flows are
modeled: in the first transformation they are modeled as (infinitely large)
queues, in the second transformation they are modeled as operations combining
the two data transformers connected by the data flow.
The advantage of the latter transformation is that a more abstract
interpretation of \DFDs\ can be achieved, because the transformation
solely focuses on modeling the information flow through a \DFD.
This is also the reason for choosing this transformation 
as the basis for the transformation described in this paper.
One simplification with respect to the transformation described in
\cite{Plat&91a} is that the latter is
more general because the order
in which the `underlying' operations are called is left unspecified
(i.e. it is loosely specified),
which makes the operation modeling the data flow rather complicated.
In this paper, however, we are dealing with purely sequential systems,
and therefore we can assume that data flows between two
data transformers are `direct' in the sense that 
the data transformer that uses the data flow as input cannot be
called before the data transformer that uses the data flow as output.

\subsection{Transformation of DFD building blocks}


When providing a formal semantics for \DFDs\ it is
important to decide whether the \DFD\ is intended to model
a concurrent system or a sequential system. More recent versions
of \SA\ (like {\em SA/RT} \cite{Ward85}) include concurrency and can be
used to develop real-time systems. However, originally \SA\
was intended for the development of information systems
implemented in traditional imperative programming languages.
In that situation it seems natural to interpret the data transformers
as {\em functions} or
{\em operations} which, given input data, sequentially perform computations and
produce output data.
If the data flow diagrams are used to model a concurrent system
it is more natural to interpret data transformers as {\em processes},
possibly executing in parallel.
Since we restrict ourselves to sequential systems we model data transformers
as {\em VDM operations}\footnote{Data transformers neither having access to
data stores nor being connected to external processes can also be modeled
as {\em VDM functions}. In our approach only {\tiny VDM}
operations are used because
we want each different type of construct in a {\tiny DFD} to be mapped to 
(semantically) the same construct in {\tiny VDM}.
{\tiny VDM} functions and {\tiny VDM} operations
(without side-effects) semantically differ in the way {\em looseness} is interpreted
(see \cite{IFIP}).}.

To ensure that the structure of the \VDM\ specification
resembles the structure of the \DFD,
we group the operations modeling data transformers at the same level in
a hierarchy of \DFDs\ together in `modules'\footnote{%
{\tiny VDM-SL} as described in \cite{BSIVDM92b} has no structuring mechanism.
The structuring mechanism we used is based on a proposal by Bear
\cite{Bear88}. The constructs we use are simple so that an intuitive
interpretation suffices.}
importing the necessary types and
operations needed for the data transformers
(figure~\ref{hierarchy}).

%\makefigure{hierarchy}{Transformation of a hierarchy of DFDs into a VDM module structure (example)}{hierarchy}


{\em External processes} can be considered as processes `executing' in parallel with
the specified system.
In our approach we model the data flows from and to external processes
as {\em state components} in the \VDM\ specification.
This is a minor difference with the transformation presented in
\cite{Plat&91a},
in which external processes are regarded as part of the system and
are therefore modeled as \VDM\ operations in the same way as data transformers.


{\em Data stores} are modeled as \VDM\ {\em state components}.
This corresponds to the fact that data transformers
(which can be used to access and change data stores)
are modeled as \VDM\ operations,
the constructs in \VDMSL\ that can access and change state components.


We envisage {\em data flows} as constructs which can combine
two data transformers by providing communication facilities between
these two data transformers.
A data flow is, therefore, modeled as an operation
calling the operations that model the two data transformers connected by
the data flow. 
In this way a process of combining data transformers can be started during
which in each step two data transformers (connected by a data flow) are
integrated into a higher level data transformer, finally resulting in
the context diagram.

Generalizing this approach, we have chosen to combine 
all the data transformers in a \DFD\ into a
higher level data transformer in {\em one} step.
The data transformer constructed in this way is modeled as a \VDM\ operation.

\subsection{Limitations imposed upon the DFDs}

Besides restricting the expressibility of the kind of \DFDs\
for which we are able to provide semantics to sequential systems, we
assume that:

\begin{itemize}
\item Data flows not connected
      to an external process must form an acyclic
      graph at each level in the hierarchy of \DFDs. This
      is necessary because in our transformation we provide both
	  explicit \VDM\ specifications as well as implicit \VDM\ specifications
	  as models for \DFDs. Allowing general cyclic \DFDs\ would make the 
	  transformation into an explicit \VDM\ specification impossible.
	  The restriction furthermore simplifies the
      transformation of \DFDs\ into implicit \VDM\ specifications.
      In Section~\ref{depend} we come back to this restriction in more
	  detail.
\item There is a one-to-one mapping between the
      input to the system and the output from the system.
	  One-to-many mappings and many-to-one mappings are a common
	  problem when interpreting \DFDs, described in more detail
	  in \cite{Alabiso88}\footnote{%
      In \cite{Alabiso88} this problem is called {\em I/O uncohesiveness}.
      I/O uncohesiveness occurs if either a data transformer
      must consume several pieces of input data before generating
      output data, or if a data transformer generates pieces of output
      independently of all other inputs and outputs.
      Alabiso describes a solution called `the burial method',
	  centered around the generation of terminator symbols
      which indicate that `something is missing'.}.
	  However, we are not entirely satisfied with the solution proposed by
	  Alabiso, and since in our experience most of the \DFDs\ with
	  one-to-many or many-to-one mappings should be regarded as design
	  products and not as specification products, we feel that a
	  restriction to one-to-one mappings is not a serious one for our
	  purpose.
      Alternatively, the analyst may supply a mini-specification for 
      each non-primitive data transformer not obeying the restriction
	  of a one-to-one mapping between input and output.

\item To simplify the formal description the
data flows must have unique names at each level in the hierarchy
of \DFDs.
\end{itemize}

\documentclass[11pt]{article}
\usepackage{a4}
\usepackage{makeidx}
%\usepackage{vdmsl-2e}
%\usepackage{color}
\usepackage{overture}
\usepackage{termref}
\usepackage{epsf}
\usepackage{latexsym}
\usepackage{longtable}

\newcommand{\StateDef}[1]{{\bf #1}}
\newcommand{\TypeDef}[1]{{\bf #1}}
\newcommand{\TypeOcc}[1]{{\it #1}}
\newcommand{\FuncDef}[1]{{\bf #1}}
\newcommand{\FuncOcc}[1]{#1}
\newcommand{\ModDef}[1]{{\tiny #1}}

\makeindex
%\documentstyle[epsf,a4,11pt,vdmsl_v1.1.31,makeidx,termref]{article}
\newcommand{\figdir}{/home/peter/toolbox}
\newcommand{\SA}{{\small SA}}
\newcommand{\SD}{{\small SD}}
\newcommand{\SASD}{{\small SA/SD}}
\newcommand{\VDM}{{\small VDM}}
\newcommand{\VDMSL}{{\small VDM-SL}}
\newcommand{\DFD}{{\small DFD}}
\newcommand{\DFDs}{{\small DFD}s}

\newcommand{\makefigure}[3]{\begin{figure}[ht]
{\leavevmode
\centering
\epsfbox{\figdir/#1.eps}
\caption{#2}\label{#3}}
\end{figure}}

\newcommand{\documenttype}{document}
\makeindex
\begin{document}

\title{A Formal Semantics of Data Flow Diagrams}
\author{Peter Gorm Larsen, Nico Plat and Hans Toetenel}
\date{November 1993}
\maketitle
\begin{abstract}
This document presents a full version of the
formal semantics of data flow diagrams reported in \cite{Larsen&93b}.
Data Flow Diagrams are used in Structured Analysis and are
 based on an abstract model for data flow transformations.
The semantics consists of a collection of \VDM\ functions,
transforming an abstract syntax representation of a data flow
diagram into an abstract syntax representation of a \VDM\
specification. Since this transformation is executable, it
becomes possible to provide a software analyst/designer with two `views' of
the system being modeled: a graphical view in terms of a data
flow diagram, and a textual view in terms of a \VDM\
specification. The specification presented in this document have been
processed by The IFAD VDM-SL Toolbox \cite{Lassen93} and the \LaTeX\ output is
produced directly by means of this tool. The complete transformation has
been syntax-checked, type-checked and tested using the
{\small IFAD VDM-SL}\ Toolbox \cite{Lassen93}; this has given us
confidence that the transformation we have defined is a reasonable one.
\end{abstract}
\newpage
\tableofcontents
\newpage
\section{Introduction}

The introduction of formal methods in industrial organizations
may become easier if these methods can be used alongside the
more widely used conventional techniques for software development,
such as `structured methods'.
Structured methods are methods for software analysis and design,
based on the use of heuristics for making analysis and design decisions.
They provide a relatively well-defined path, often in a cookbook-like
fashion (hence the term `structured' methods), starting from the analysis
of software requirements and ending at system coding.
The design notations used are usually graphical and have no formal basis.
In that sense structured and formal methods can be regarded as complementary.
It is often suggested that the informal graphical notations as provided by
structured methods are intuitively appealing to software analysts/designers.
Therefore, a combined structured/formal method may not only increase the
understanding of the use of formal methods in the software process, but 
also may increase the acceptability of formal methods to these people.

Our work in this area so far has concentrated on combining {\em Structured
Analysis (SA)} \cite{Yourdon75,Demarco78,Gane77}
with the {\em Vienna Development Method (VDM)}
\cite{Bjorner&82b,Jones90a}; we provide a brief introduction to
\SA, but we refer to text books such as \cite{Jones90a} and
\cite{Andrews&91} for an introduction to \VDM.
We think that a well-integrated combination of notations can be achieved
by using {\em data flow diagrams (DFDs)} -- which we consider to be the main
design notation of \SA\ --
as a graphical view of the system and \VDM\ as a textual
view. These different views emphasize different aspects of the
specified system: the \DFD\ graphical view
focuses on an overview of the {\em structure} of the system, whereas
the \VDM\ textual view focuses on the detailed {\em functionality} of the system.
The base of a combined structured/formal method consists of a formally defined
relation between the structured method and the formal method.
In \cite{Plat&91a} we describe several approaches to modeling \DFDs\ 
using the \VDMSL\ specification language \cite{BSIVDM92b,Dawes91}.
In this paper we discuss one such particular model in more detail,
thus essentially providing a `formal semantics' of \DFDs.
A discussion on the methodological aspects of the approach can be found in
\cite{Larsen&91b}.

The remainder of this paper is organized as follows.
In the following section a brief introduction to \SA\ is given, focusing on the
use of \DFDs.
In Section~\ref{sec:motivation} we describe our strategy for
transforming \DFDs\ into \VDM\ specifications, paying attention to the
limitations of our approach.
The main part of this paper is Section~\ref{transsec}, in which the formal
transformation from \DFDs\ to \VDM\ is presented.
Emphasis is put on the motivation for the choices
made in the transformation. The main aspects of the transformation
itself are described using \VDM\ functions together
with a number of examples.
In Section~\ref{sec:conclusions} we give an overview of related
work on formal semantics for \DFDs, and present some conclusions and
ideas for future work in this area.
This is followed by a list of references, and two appendices and an
index for the complete specification.
In the first appendix, we describe the abstract syntax representation
of \DFDs. In the second appendix we present the abstract syntax 
representation we use for the generated
\VDM\ specifications. This is a subset of the one used
in \cite{BSIVDM92b}\footnote{%
To be precise, the abstract syntax used for \VDM\ specifications is
a partof the one
called the `Outer Abstract Syntax' in \cite{BSIVDM92b}; a lack of knowledge about
this Outer Abstract Syntax
does not affect the understanding of this document, however.}).

\section{Overview of Structured Analysis}
\label{sec:sa}

{\em Structured Analysis ({\small\it SA})}
\cite{Yourdon75,Demarco78,Gane77} is one of
the most widely used methods for software analysis.
Often it is used in combination with {\em Structured Design ({\small\it SD})}
\cite{Yourdon75}; the resulting combination is called {\em \SASD}.
The approach to analysis taken in \SA\ is to concentrate on the functions
to be carried out by the system, using data flow abstraction to
describe the flow of data through a network of transforming
processes, called data transformers, together with access to data stores.
Such a network, which is the most important design product of \SA,
is called a {\em data flow diagram (DFD)}. The original version of
\SA\ was meant to be used to model sequential systems.  
A \DFD\ is a {\em directed graph} consisting of elementary building blocks.
Each building block has a graphical notation (figure~\ref{figexternalprocess}).

%\makefigure{building_blocks}{Elementary building blocks of a DFD}{figexternalprocess}

Through the years several dialects have evolved and extensions
have been defined (e.g. {\small SSADM}\ \cite{Longworth&86} and {\small SA/RT}\
\cite{Ward85}), 
but we limit ourselves to \DFDs\ with a sequential model and
a small number of building blocks:
\begin{itemize}
\item
{\em Data transformers}. Data transformers denote a transformation from
(an arbitrary number of) input values to
(an arbitrary number of) output values, possibly with side effects.
\item
{\em Data flows}. Data flows are represented as arrows, connecting one
data transformer to another. They represent a flow of data between the
data transformers they connect.
The flow of data is unidirectional in the direction of the arrow.
\item
{\em Data stores}. Data stores provide for (temporary) storage of data.
\item
{\em External processes.}
External processes are processes that are not part of the system but belong
to the outside world. They are used to show where the input to the system
is coming from and where the output of the system is going to.
\end{itemize}

\DFD s are used to model the {\em information flow} through a system.
As such they provide a limited view of the system:
in their most rudimentary form they neither show the control flow of the system
nor any timing aspects.
Therefore, \DFDs\ are often combined with data dictionaries,
control flow diagrams, state transition diagrams, decision tables
and mini-specifications to provide a comprehensive view of all the 
aspects of the system.

The process of constructing a \DFD\ is an iterative process.
Initially, the system to be designed is envisaged as one large data transformer,
getting input from and providing output to external processes.
This initial, high-level \DFD\ is called a {\em context diagram}.
The next step is the {\em decomposition}
of the context diagram into a network of data transformers,
the total network providing the same functionality as the original
context diagram.
This process is repeated for each data transformer until the analyst/designer
considers all the data transformers in the \DFD\ to be primitive,
i.e. each data transformer performs a simple operation that does not
need to be further decomposed.
We call such a collection of \DFDs, describing the same system but at different
levels of abstraction, a {\em hierarchy of DFDs}.

\section{Approach to the Transformation}
\label{sec:motivation}

Before presenting the formal transformation from \DFDs\ to \VDM\ we first explain
the underlying strategy for the transformation and the limitations
imposed upon the \DFDs\ to make our transformation valid.

\subsection{Underlying strategy}

The starting point for our transformation is the work presented in
\cite{Plat&91a}, in which the general properties of two transformations from
DFDs to \VDM\ constructs are discussed.
The main difference between these two transformations is the way data flows are
modeled: in the first transformation they are modeled as (infinitely large)
queues, in the second transformation they are modeled as operations combining
the two data transformers connected by the data flow.
The advantage of the latter transformation is that a more abstract
interpretation of \DFDs\ can be achieved, because the transformation
solely focuses on modeling the information flow through a \DFD.
This is also the reason for choosing this transformation 
as the basis for the transformation described in this paper.
One simplification with respect to the transformation described in
\cite{Plat&91a} is that the latter is
more general because the order
in which the `underlying' operations are called is left unspecified
(i.e. it is loosely specified),
which makes the operation modeling the data flow rather complicated.
In this paper, however, we are dealing with purely sequential systems,
and therefore we can assume that data flows between two
data transformers are `direct' in the sense that 
the data transformer that uses the data flow as input cannot be
called before the data transformer that uses the data flow as output.

\subsection{Transformation of DFD building blocks}


When providing a formal semantics for \DFDs\ it is
important to decide whether the \DFD\ is intended to model
a concurrent system or a sequential system. More recent versions
of \SA\ (like {\em SA/RT} \cite{Ward85}) include concurrency and can be
used to develop real-time systems. However, originally \SA\
was intended for the development of information systems
implemented in traditional imperative programming languages.
In that situation it seems natural to interpret the data transformers
as {\em functions} or
{\em operations} which, given input data, sequentially perform computations and
produce output data.
If the data flow diagrams are used to model a concurrent system
it is more natural to interpret data transformers as {\em processes},
possibly executing in parallel.
Since we restrict ourselves to sequential systems we model data transformers
as {\em VDM operations}\footnote{Data transformers neither having access to
data stores nor being connected to external processes can also be modeled
as {\em VDM functions}. In our approach only {\tiny VDM}
operations are used because
we want each different type of construct in a {\tiny DFD} to be mapped to 
(semantically) the same construct in {\tiny VDM}.
{\tiny VDM} functions and {\tiny VDM} operations
(without side-effects) semantically differ in the way {\em looseness} is interpreted
(see \cite{IFIP}).}.

To ensure that the structure of the \VDM\ specification
resembles the structure of the \DFD,
we group the operations modeling data transformers at the same level in
a hierarchy of \DFDs\ together in `modules'\footnote{%
{\tiny VDM-SL} as described in \cite{BSIVDM92b} has no structuring mechanism.
The structuring mechanism we used is based on a proposal by Bear
\cite{Bear88}. The constructs we use are simple so that an intuitive
interpretation suffices.}
importing the necessary types and
operations needed for the data transformers
(figure~\ref{hierarchy}).

%\makefigure{hierarchy}{Transformation of a hierarchy of DFDs into a VDM module structure (example)}{hierarchy}


{\em External processes} can be considered as processes `executing' in parallel with
the specified system.
In our approach we model the data flows from and to external processes
as {\em state components} in the \VDM\ specification.
This is a minor difference with the transformation presented in
\cite{Plat&91a},
in which external processes are regarded as part of the system and
are therefore modeled as \VDM\ operations in the same way as data transformers.


{\em Data stores} are modeled as \VDM\ {\em state components}.
This corresponds to the fact that data transformers
(which can be used to access and change data stores)
are modeled as \VDM\ operations,
the constructs in \VDMSL\ that can access and change state components.


We envisage {\em data flows} as constructs which can combine
two data transformers by providing communication facilities between
these two data transformers.
A data flow is, therefore, modeled as an operation
calling the operations that model the two data transformers connected by
the data flow. 
In this way a process of combining data transformers can be started during
which in each step two data transformers (connected by a data flow) are
integrated into a higher level data transformer, finally resulting in
the context diagram.

Generalizing this approach, we have chosen to combine 
all the data transformers in a \DFD\ into a
higher level data transformer in {\em one} step.
The data transformer constructed in this way is modeled as a \VDM\ operation.

\subsection{Limitations imposed upon the DFDs}

Besides restricting the expressibility of the kind of \DFDs\
for which we are able to provide semantics to sequential systems, we
assume that:

\begin{itemize}
\item Data flows not connected
      to an external process must form an acyclic
      graph at each level in the hierarchy of \DFDs. This
      is necessary because in our transformation we provide both
	  explicit \VDM\ specifications as well as implicit \VDM\ specifications
	  as models for \DFDs. Allowing general cyclic \DFDs\ would make the 
	  transformation into an explicit \VDM\ specification impossible.
	  The restriction furthermore simplifies the
      transformation of \DFDs\ into implicit \VDM\ specifications.
      In Section~\ref{depend} we come back to this restriction in more
	  detail.
\item There is a one-to-one mapping between the
      input to the system and the output from the system.
	  One-to-many mappings and many-to-one mappings are a common
	  problem when interpreting \DFDs, described in more detail
	  in \cite{Alabiso88}\footnote{%
      In \cite{Alabiso88} this problem is called {\em I/O uncohesiveness}.
      I/O uncohesiveness occurs if either a data transformer
      must consume several pieces of input data before generating
      output data, or if a data transformer generates pieces of output
      independently of all other inputs and outputs.
      Alabiso describes a solution called `the burial method',
	  centered around the generation of terminator symbols
      which indicate that `something is missing'.}.
	  However, we are not entirely satisfied with the solution proposed by
	  Alabiso, and since in our experience most of the \DFDs\ with
	  one-to-many or many-to-one mappings should be regarded as design
	  products and not as specification products, we feel that a
	  restriction to one-to-one mappings is not a serious one for our
	  purpose.
      Alternatively, the analyst may supply a mini-specification for 
      each non-primitive data transformer not obeying the restriction
	  of a one-to-one mapping between input and output.

\item To simplify the formal description the
data flows must have unique names at each level in the hierarchy
of \DFDs.
\end{itemize}

\documentclass[11pt]{article}
\usepackage{a4}
\usepackage{makeidx}
%\usepackage{vdmsl-2e}
%\usepackage{color}
\usepackage{overture}
\usepackage{termref}
\usepackage{epsf}
\usepackage{latexsym}
\usepackage{longtable}

\newcommand{\StateDef}[1]{{\bf #1}}
\newcommand{\TypeDef}[1]{{\bf #1}}
\newcommand{\TypeOcc}[1]{{\it #1}}
\newcommand{\FuncDef}[1]{{\bf #1}}
\newcommand{\FuncOcc}[1]{#1}
\newcommand{\ModDef}[1]{{\tiny #1}}

\makeindex
%\documentstyle[epsf,a4,11pt,vdmsl_v1.1.31,makeidx,termref]{article}
\newcommand{\figdir}{/home/peter/toolbox}
\newcommand{\SA}{{\small SA}}
\newcommand{\SD}{{\small SD}}
\newcommand{\SASD}{{\small SA/SD}}
\newcommand{\VDM}{{\small VDM}}
\newcommand{\VDMSL}{{\small VDM-SL}}
\newcommand{\DFD}{{\small DFD}}
\newcommand{\DFDs}{{\small DFD}s}

\newcommand{\makefigure}[3]{\begin{figure}[ht]
{\leavevmode
\centering
\epsfbox{\figdir/#1.eps}
\caption{#2}\label{#3}}
\end{figure}}

\newcommand{\documenttype}{document}
\makeindex
\begin{document}

\title{A Formal Semantics of Data Flow Diagrams}
\author{Peter Gorm Larsen, Nico Plat and Hans Toetenel}
\date{November 1993}
\maketitle
\begin{abstract}
This document presents a full version of the
formal semantics of data flow diagrams reported in \cite{Larsen&93b}.
Data Flow Diagrams are used in Structured Analysis and are
 based on an abstract model for data flow transformations.
The semantics consists of a collection of \VDM\ functions,
transforming an abstract syntax representation of a data flow
diagram into an abstract syntax representation of a \VDM\
specification. Since this transformation is executable, it
becomes possible to provide a software analyst/designer with two `views' of
the system being modeled: a graphical view in terms of a data
flow diagram, and a textual view in terms of a \VDM\
specification. The specification presented in this document have been
processed by The IFAD VDM-SL Toolbox \cite{Lassen93} and the \LaTeX\ output is
produced directly by means of this tool. The complete transformation has
been syntax-checked, type-checked and tested using the
{\small IFAD VDM-SL}\ Toolbox \cite{Lassen93}; this has given us
confidence that the transformation we have defined is a reasonable one.
\end{abstract}
\newpage
\tableofcontents
\newpage
\section{Introduction}

The introduction of formal methods in industrial organizations
may become easier if these methods can be used alongside the
more widely used conventional techniques for software development,
such as `structured methods'.
Structured methods are methods for software analysis and design,
based on the use of heuristics for making analysis and design decisions.
They provide a relatively well-defined path, often in a cookbook-like
fashion (hence the term `structured' methods), starting from the analysis
of software requirements and ending at system coding.
The design notations used are usually graphical and have no formal basis.
In that sense structured and formal methods can be regarded as complementary.
It is often suggested that the informal graphical notations as provided by
structured methods are intuitively appealing to software analysts/designers.
Therefore, a combined structured/formal method may not only increase the
understanding of the use of formal methods in the software process, but 
also may increase the acceptability of formal methods to these people.

Our work in this area so far has concentrated on combining {\em Structured
Analysis (SA)} \cite{Yourdon75,Demarco78,Gane77}
with the {\em Vienna Development Method (VDM)}
\cite{Bjorner&82b,Jones90a}; we provide a brief introduction to
\SA, but we refer to text books such as \cite{Jones90a} and
\cite{Andrews&91} for an introduction to \VDM.
We think that a well-integrated combination of notations can be achieved
by using {\em data flow diagrams (DFDs)} -- which we consider to be the main
design notation of \SA\ --
as a graphical view of the system and \VDM\ as a textual
view. These different views emphasize different aspects of the
specified system: the \DFD\ graphical view
focuses on an overview of the {\em structure} of the system, whereas
the \VDM\ textual view focuses on the detailed {\em functionality} of the system.
The base of a combined structured/formal method consists of a formally defined
relation between the structured method and the formal method.
In \cite{Plat&91a} we describe several approaches to modeling \DFDs\ 
using the \VDMSL\ specification language \cite{BSIVDM92b,Dawes91}.
In this paper we discuss one such particular model in more detail,
thus essentially providing a `formal semantics' of \DFDs.
A discussion on the methodological aspects of the approach can be found in
\cite{Larsen&91b}.

The remainder of this paper is organized as follows.
In the following section a brief introduction to \SA\ is given, focusing on the
use of \DFDs.
In Section~\ref{sec:motivation} we describe our strategy for
transforming \DFDs\ into \VDM\ specifications, paying attention to the
limitations of our approach.
The main part of this paper is Section~\ref{transsec}, in which the formal
transformation from \DFDs\ to \VDM\ is presented.
Emphasis is put on the motivation for the choices
made in the transformation. The main aspects of the transformation
itself are described using \VDM\ functions together
with a number of examples.
In Section~\ref{sec:conclusions} we give an overview of related
work on formal semantics for \DFDs, and present some conclusions and
ideas for future work in this area.
This is followed by a list of references, and two appendices and an
index for the complete specification.
In the first appendix, we describe the abstract syntax representation
of \DFDs. In the second appendix we present the abstract syntax 
representation we use for the generated
\VDM\ specifications. This is a subset of the one used
in \cite{BSIVDM92b}\footnote{%
To be precise, the abstract syntax used for \VDM\ specifications is
a partof the one
called the `Outer Abstract Syntax' in \cite{BSIVDM92b}; a lack of knowledge about
this Outer Abstract Syntax
does not affect the understanding of this document, however.}).

\section{Overview of Structured Analysis}
\label{sec:sa}

{\em Structured Analysis ({\small\it SA})}
\cite{Yourdon75,Demarco78,Gane77} is one of
the most widely used methods for software analysis.
Often it is used in combination with {\em Structured Design ({\small\it SD})}
\cite{Yourdon75}; the resulting combination is called {\em \SASD}.
The approach to analysis taken in \SA\ is to concentrate on the functions
to be carried out by the system, using data flow abstraction to
describe the flow of data through a network of transforming
processes, called data transformers, together with access to data stores.
Such a network, which is the most important design product of \SA,
is called a {\em data flow diagram (DFD)}. The original version of
\SA\ was meant to be used to model sequential systems.  
A \DFD\ is a {\em directed graph} consisting of elementary building blocks.
Each building block has a graphical notation (figure~\ref{figexternalprocess}).

%\makefigure{building_blocks}{Elementary building blocks of a DFD}{figexternalprocess}

Through the years several dialects have evolved and extensions
have been defined (e.g. {\small SSADM}\ \cite{Longworth&86} and {\small SA/RT}\
\cite{Ward85}), 
but we limit ourselves to \DFDs\ with a sequential model and
a small number of building blocks:
\begin{itemize}
\item
{\em Data transformers}. Data transformers denote a transformation from
(an arbitrary number of) input values to
(an arbitrary number of) output values, possibly with side effects.
\item
{\em Data flows}. Data flows are represented as arrows, connecting one
data transformer to another. They represent a flow of data between the
data transformers they connect.
The flow of data is unidirectional in the direction of the arrow.
\item
{\em Data stores}. Data stores provide for (temporary) storage of data.
\item
{\em External processes.}
External processes are processes that are not part of the system but belong
to the outside world. They are used to show where the input to the system
is coming from and where the output of the system is going to.
\end{itemize}

\DFD s are used to model the {\em information flow} through a system.
As such they provide a limited view of the system:
in their most rudimentary form they neither show the control flow of the system
nor any timing aspects.
Therefore, \DFDs\ are often combined with data dictionaries,
control flow diagrams, state transition diagrams, decision tables
and mini-specifications to provide a comprehensive view of all the 
aspects of the system.

The process of constructing a \DFD\ is an iterative process.
Initially, the system to be designed is envisaged as one large data transformer,
getting input from and providing output to external processes.
This initial, high-level \DFD\ is called a {\em context diagram}.
The next step is the {\em decomposition}
of the context diagram into a network of data transformers,
the total network providing the same functionality as the original
context diagram.
This process is repeated for each data transformer until the analyst/designer
considers all the data transformers in the \DFD\ to be primitive,
i.e. each data transformer performs a simple operation that does not
need to be further decomposed.
We call such a collection of \DFDs, describing the same system but at different
levels of abstraction, a {\em hierarchy of DFDs}.

\section{Approach to the Transformation}
\label{sec:motivation}

Before presenting the formal transformation from \DFDs\ to \VDM\ we first explain
the underlying strategy for the transformation and the limitations
imposed upon the \DFDs\ to make our transformation valid.

\subsection{Underlying strategy}

The starting point for our transformation is the work presented in
\cite{Plat&91a}, in which the general properties of two transformations from
DFDs to \VDM\ constructs are discussed.
The main difference between these two transformations is the way data flows are
modeled: in the first transformation they are modeled as (infinitely large)
queues, in the second transformation they are modeled as operations combining
the two data transformers connected by the data flow.
The advantage of the latter transformation is that a more abstract
interpretation of \DFDs\ can be achieved, because the transformation
solely focuses on modeling the information flow through a \DFD.
This is also the reason for choosing this transformation 
as the basis for the transformation described in this paper.
One simplification with respect to the transformation described in
\cite{Plat&91a} is that the latter is
more general because the order
in which the `underlying' operations are called is left unspecified
(i.e. it is loosely specified),
which makes the operation modeling the data flow rather complicated.
In this paper, however, we are dealing with purely sequential systems,
and therefore we can assume that data flows between two
data transformers are `direct' in the sense that 
the data transformer that uses the data flow as input cannot be
called before the data transformer that uses the data flow as output.

\subsection{Transformation of DFD building blocks}


When providing a formal semantics for \DFDs\ it is
important to decide whether the \DFD\ is intended to model
a concurrent system or a sequential system. More recent versions
of \SA\ (like {\em SA/RT} \cite{Ward85}) include concurrency and can be
used to develop real-time systems. However, originally \SA\
was intended for the development of information systems
implemented in traditional imperative programming languages.
In that situation it seems natural to interpret the data transformers
as {\em functions} or
{\em operations} which, given input data, sequentially perform computations and
produce output data.
If the data flow diagrams are used to model a concurrent system
it is more natural to interpret data transformers as {\em processes},
possibly executing in parallel.
Since we restrict ourselves to sequential systems we model data transformers
as {\em VDM operations}\footnote{Data transformers neither having access to
data stores nor being connected to external processes can also be modeled
as {\em VDM functions}. In our approach only {\tiny VDM}
operations are used because
we want each different type of construct in a {\tiny DFD} to be mapped to 
(semantically) the same construct in {\tiny VDM}.
{\tiny VDM} functions and {\tiny VDM} operations
(without side-effects) semantically differ in the way {\em looseness} is interpreted
(see \cite{IFIP}).}.

To ensure that the structure of the \VDM\ specification
resembles the structure of the \DFD,
we group the operations modeling data transformers at the same level in
a hierarchy of \DFDs\ together in `modules'\footnote{%
{\tiny VDM-SL} as described in \cite{BSIVDM92b} has no structuring mechanism.
The structuring mechanism we used is based on a proposal by Bear
\cite{Bear88}. The constructs we use are simple so that an intuitive
interpretation suffices.}
importing the necessary types and
operations needed for the data transformers
(figure~\ref{hierarchy}).

%\makefigure{hierarchy}{Transformation of a hierarchy of DFDs into a VDM module structure (example)}{hierarchy}


{\em External processes} can be considered as processes `executing' in parallel with
the specified system.
In our approach we model the data flows from and to external processes
as {\em state components} in the \VDM\ specification.
This is a minor difference with the transformation presented in
\cite{Plat&91a},
in which external processes are regarded as part of the system and
are therefore modeled as \VDM\ operations in the same way as data transformers.


{\em Data stores} are modeled as \VDM\ {\em state components}.
This corresponds to the fact that data transformers
(which can be used to access and change data stores)
are modeled as \VDM\ operations,
the constructs in \VDMSL\ that can access and change state components.


We envisage {\em data flows} as constructs which can combine
two data transformers by providing communication facilities between
these two data transformers.
A data flow is, therefore, modeled as an operation
calling the operations that model the two data transformers connected by
the data flow. 
In this way a process of combining data transformers can be started during
which in each step two data transformers (connected by a data flow) are
integrated into a higher level data transformer, finally resulting in
the context diagram.

Generalizing this approach, we have chosen to combine 
all the data transformers in a \DFD\ into a
higher level data transformer in {\em one} step.
The data transformer constructed in this way is modeled as a \VDM\ operation.

\subsection{Limitations imposed upon the DFDs}

Besides restricting the expressibility of the kind of \DFDs\
for which we are able to provide semantics to sequential systems, we
assume that:

\begin{itemize}
\item Data flows not connected
      to an external process must form an acyclic
      graph at each level in the hierarchy of \DFDs. This
      is necessary because in our transformation we provide both
	  explicit \VDM\ specifications as well as implicit \VDM\ specifications
	  as models for \DFDs. Allowing general cyclic \DFDs\ would make the 
	  transformation into an explicit \VDM\ specification impossible.
	  The restriction furthermore simplifies the
      transformation of \DFDs\ into implicit \VDM\ specifications.
      In Section~\ref{depend} we come back to this restriction in more
	  detail.
\item There is a one-to-one mapping between the
      input to the system and the output from the system.
	  One-to-many mappings and many-to-one mappings are a common
	  problem when interpreting \DFDs, described in more detail
	  in \cite{Alabiso88}\footnote{%
      In \cite{Alabiso88} this problem is called {\em I/O uncohesiveness}.
      I/O uncohesiveness occurs if either a data transformer
      must consume several pieces of input data before generating
      output data, or if a data transformer generates pieces of output
      independently of all other inputs and outputs.
      Alabiso describes a solution called `the burial method',
	  centered around the generation of terminator symbols
      which indicate that `something is missing'.}.
	  However, we are not entirely satisfied with the solution proposed by
	  Alabiso, and since in our experience most of the \DFDs\ with
	  one-to-many or many-to-one mappings should be regarded as design
	  products and not as specification products, we feel that a
	  restriction to one-to-one mappings is not a serious one for our
	  purpose.
      Alternatively, the analyst may supply a mini-specification for 
      each non-primitive data transformer not obeying the restriction
	  of a one-to-one mapping between input and output.

\item To simplify the formal description the
data flows must have unique names at each level in the hierarchy
of \DFDs.
\end{itemize}

\include{generated/latex/specification/dfdexample.vdmsl}

\newpage
\bibliographystyle{abbrv}
\bibliography{savdm} 

\addcontentsline{toc}{section}{Index}
\printindex

\end{document}
  


\newpage
\bibliographystyle{abbrv}
\bibliography{savdm} 

\addcontentsline{toc}{section}{Index}
\printindex

\end{document}
  


\newpage
\bibliographystyle{abbrv}
\bibliography{savdm} 

\addcontentsline{toc}{section}{Index}
\printindex

\end{document}
  


\newpage
\bibliographystyle{abbrv}
\bibliography{savdm} 

\addcontentsline{toc}{section}{Index}
\printindex

\end{document}
  
